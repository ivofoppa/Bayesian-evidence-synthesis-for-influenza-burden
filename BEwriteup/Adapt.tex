\documentclass{beamer}
\mode<presentation>
{
  \usetheme{Singapore}
  \usecolortheme{wolverine}
%  \setbeamercovered{transparent}
  \setbeamercovered{invisible}
 }
\usepackage[english]{babel}
\usepackage[latin1]{inputenc}
\usepackage{graphicx}
\usepackage{times}
\usepackage[T1]{fontenc}
\usepackage{mathrsfs}
\usepackage{listings}
\usepackage{verbatim}
\usepackage{color, colortbl}
\usepackage{soul}
 \usepackage{tkz-graph}  
% \usetikzlibrary{positioning} 
\usetikzlibrary{shapes.geometric}%   
\newcommand\SoulColor{%
	\let\set@color\beamerorig@set@color
	\let\reset@color\beamerorig@reset@color}
\newcommand*{\Scale}[2][4]{\scalebox{#1}{$#2$}}%

\lstset{
	numberstyle=\footnotesize,
	basicstyle=\ttfamily\tiny}
%%%%%%%%%%%%%%%%%%%%%%%%%%%%%%%%%%%%%%%%%%%%%%%%%%%%%%%%%%%%%%%5
\title[TND] % (optional, use only with long paper titles)
{Confounding of Influenza VE Estimates by History of Vaccination}
%\subtitle
%{An attempt to vindicate the 'Bayesian approach'}
\author{Ivo M. Foppa}
\date % (optional, should be abbreviation of conference name)
{5/5/2017}
% If you wish to uncover everything in a step-wise fashion, uncomment
% the following command:
%\beamerdefaultoverlayspecification{<+->}
\begin{document}
\begin{frame}
  \titlepage
\end{frame}
\section{Background}
%
\begin{frame}
	
	{\small Ohmit et al. (2014). Influenza vaccine effectiveness in the 2011-2012 season: protection against each circulating virus and the effect of prior vaccination on estimates. \emph{Clinical Infectious Diseases}, 58(3), 319-327.}
	\begin{figure}
		\includegraphics[width = \linewidth]{table1.png}
	\end{figure}
\end{frame}
%
\begin{frame}
	{Could the apparently deleterious effect of repeated vaccinations on VE be due to confounding by vaccination history?}
	\begin{itemize}
		\item If vaccination reduces natural infection due to influenza, those unvaccinateed will have a higher incidence of natural infection, possibly protecting them from future infection (``infection block'' hypothesis)
		%
		\item People tend to either get vaccinated every season or not at all (or sometimes); thus vaccination in the current season is correlated with vaccination history.
	\end{itemize} 
\end{frame}
%
\begin{frame}{Confounding by vaccination history?}
	%\setbeamercovered{invisible}
	
	{\Large	\begin {tikzpicture}[>= stealth]
		
		\node (v)[circle,line width=1.5,minimum size=2cm,draw] at (0,5) {$v_k$};
		
		\node (R)[circle,line width=1.5,minimum size=2cm,draw] at (9,5) {$R_k$};
		\draw[->,line width=1.5,dashed,red](v) edge node [above] {???} (R);
		
		\visible<2->{\node (vhist)[circle,line width=1.5,minimum size=2cm,draw] at (4.5,7.5)  {$\overline{v}^{\mathcal{S}_{k-1}}$};}
		
		\visible<3->{\draw[->,line width=1.5](vhist) edge (v) (vhist) edge  (R);}
		\end {tikzpicture}}
\end{frame}
%
\begin{frame}
	{Could the apparently deleterious effect of repeated vaccinations on VE be due to confounding by vaccination history?}
	
	\centering
	\visible<2->{\Large \textbf{\textcolor{blue}{Probably not ...}}}
	\visible<3->{\Large \textbf{\textcolor{red}{???}}}
\end{frame}

%
\section{Goal}
\begin{frame}{Goals of the study}
	The study has two goals:
	%
	\begin{enumerate}
		\item To determine, in general terms, under what circumstances confounding of VE estimates by vaccination history may be expected (mathematical argument).
		%
		\item To explore the resulting bias under a wide range of assumptions, including extreme assumptions that would favor confounding (simulation study).
	\end{enumerate}
\end{frame}
%
\section{Theory}
\begin{frame}{Theoretical foundation: Conditions for confounding}
\textbf{IF}
	\begin{enumerate}
		\item The risk of influenza infection in season $k$ is affected by the history of  vaccination up to season $k-1$, i.e. 
		%
		\begin{equation}
		\label{eq:conf_cond1}
		R_{k|\overline{v}^{\mathcal{S}_{k-1}}_a, v_k=v} \neq R_{k|\overline{v}^{\mathcal{S}_{k-1}}_b,v_k=v}
		\end{equation}
		%
		(e.g. infection block hypothesis true), 
		%
		\item[]\textbf{AND IF}
%
\item Vaccination history is serially correlated, i.e.
 \begin{equation}
		 \Pr(v_{k}=1|\overline{v}^{\mathcal{S}_{k-1}}_a) \neq \Pr(v_{k}=1|\overline{v}^{\mathcal{S}_{k-1}}_b)
 \end{equation}
	\end{enumerate}
\textbf{THEN VE estimates \emph{may} be confounded by vaccination history if that  is not adjusted for.}
\end{frame}
\section{Simulation study}
\begin{frame}{How \emph{much} confounding?}
	Simulation studies, over five ``seasons'':
	%
	{\footnotesize \begin{enumerate}
	\item \textbf{\emph{Population}}:
	People who \textbf{\emph{always}} get vaccinated (800,000); people who \textbf{\emph{never}} get vaccinated (800,000); people who used to get always vaccinated, but are \textbf{\emph{not vaccinated in current season}} (200,000); people who used to never get always vaccinated, but are \textbf{\emph{vaccinated in current season}} (200,000).
	\item \textbf{\emph{Exposure}} to influenza constant and independent of vaccination: Seasonal attack rate 80 per 1,000 or 160 per 1,000; non-influenza ARI double that.
	\item \textbf{\emph{Effect of vaccination}}: a percentage $VE$ of those vaccinated who were susceptible before are \emph{fully protected in current season} while the rest remains unaffected by vaccination (``all--or--none'').
		\item \textbf{\emph{Effect of natural influenza infection}}: Two scenarios:
	\begin{enumerate}{\scriptsize 
			\item \textbf{Scenario A}: A proportion $\gamma$ (.25 to 1.0) of those infected during the previous season remain fully immune in the next season while the rest is again susceptible,
			\item \textbf{Scenario B}: A proportion $\gamma$ (0.25 to 1.0) of those immune due to natural infection remain fully immune in the next season while the rest is again susceptible.}
	%
		\end{enumerate}
	\end{enumerate}}
\end{frame}
 %
 \subsection{Simulation results}
 \begin{frame}{Moderate incidence setting (seasonal AR 8 per 100)}
 
 {\footnotesize
 	\textbf{Scenario A: 10,000 simulations per $\gamma$ value, all potential cases/controls enrolled.}
 \begin{table}
 		%
 	\begin{tabular}{rcccc}
 		$\gamma$ & Crude VE (\%) & Crude Bias & VE, adj.  (\%)& Adj. Bias \\
 		\hline
 	0.25 & 49.7 & -0.3 & 50 & 0 \\ 
 	0.5 & 49.4 & -0.6 & 50 & 0 \\ 
 	0.75 & 49.1 & -0.9 & 50 & 0 \\ 
 	1 & 48.9 & -1.1 & 50 & 0 \\ 
 		\hline
 %		
 	\end{tabular}
 \end{table}
 %
 
 
\textbf{Scenario B}
  \begin{table}
 	\begin{tabular}{rcccc}
 		$\gamma$ & Crude VE (\%) & Crude Bias & VE, adj.  (\%)& Adj. Bias \\
 		\hline
0.25 & 49.4 & -0.6 & 50 & 0 \\ 
0.5 & 48.6 & -1.4 & 50 & 0 \\ 
0.75 & 47.2 & -2.8 & 50 & 0 \\ 
1 & 44.5 & -5.5 & 50 & 0 \\
 		\hline
 %		
 	\end{tabular}
 \end{table}
}
\end{frame}
%
 %
\begin{frame}{High incidence setting (seasonal AR 16 per 100)}

{\footnotesize
	\textbf{Scenario A:}
	\begin{table}
		%
		\begin{tabular}{rcccc}
			$\gamma$ & Crude VE (\%) & Crude Bias & VE, adj.  (\%)& Adj. Bias \\
			\hline
0.25 & 49.4 & -0.6 & 50 & 0 \\ 
0.5 & 48.9 & -1.1 & 50 & 0 \\ 
0.75 & 48.3 & -1.7 & 50 & 0 \\ 
1 & 47.8 & -2.2 & 50 & 0 \\ 
			\hline
			%		
		\end{tabular}
	\end{table}
	%
	
	
	\textbf{Scenario B: }
	\begin{table}
		\begin{tabular}{rcccc}
			$\gamma$ & Crude VE (\%) & Crude Bias & VE, adj.  (\%)& Adj. Bias \\
			\hline
0.25 & 48.9 & -1.1 & 50 & 0 \\ 
0.5 & 47.2 & -2.8 & 50 & 0 \\ 
0.75 & 44.4 & -5.6 & 50 & 0 \\ 
1 & 37.7 & -12.3 & 50 & 0 \\ 
			\hline
			%		
		\end{tabular}
	\end{table}
}
\end{frame}
%
\begin{frame}
		\centering
	\visible<2->{\Large \textbf{\textcolor{blue}{What if vaccination has a lingering effect, too?}}}
	
\end{frame}
%
\begin{frame}{Moderate incidence setting (seasonal AR 8 per 100), residual vaccination effect ($\epsilon$)}
	
	{\footnotesize
		\textbf{Only residual vaccination effect ($\epsilon$), no residual effect of natural infection ($\gamma=0$), scenario B.}
		\begin{table}
			%
			\begin{tabular}{rcccc}
				$\epsilon$ & Crude VE (\%) & Crude Bias & VE, adj.  (\%)& Adj. Bias \\
				\hline
				0.25 & 53.4 & 3.4 & 50 & 0 \\ 
				0.5 & 56.2 & 6.2 & 50 & 0 \\ 
				0.75 & 58.5 & 8.5 & 50 & 0 \\ 
				1 & 60 & 10 & 50 & 0 \\
				\hline
				%		
			\end{tabular}
		\end{table}
		%
		
		
		\textbf{Residual vaccination effect ($\epsilon$) and  residual effect of natural infection ($\gamma=0.5$), scenario B.}
		\begin{table}
			\begin{tabular}{rcccc}
				$\epsilon$ & Crude VE (\%) & Crude Bias & VE, adj.  (\%)& Adj. Bias \\
				\hline
				0.25 & 52.8 & 2.8 & 50 & 0 \\ 
				0.5 & 55.6 & 5.6 & 50 & 0 \\ 
				0.75 & 57.8 & 7.8 & 50 & 0 \\ 
				1 & 59.2 & 9.2 & 50 & 0 \\ 
				\hline
				%		
			\end{tabular}
		\end{table}
	}
\end{frame}
%
 \begin{frame}{VE, analyzing as Ohmit et al.}
	
	{\footnotesize
		\textbf{Moderate incidence setting (seasonal AR 8 per 100), scenario B.}
		\begin{table}
			%
			\begin{tabular}{rcccc}
				$\gamma$ & VE new (\%) & VE prior & VE both & VE, adj.  (\%) \\
				\hline
				0.25 & 50 & -1.3 & 49.3 & 50 \\ 
				0.5 & 50 & -3.7 & 48.1 & 50 \\ 
				0.75 & 50 & -8.4 & 45.8 & 50 \\ 
				1 & 50 & -18.6 & 40.7 & 50 \\ 
				\hline
				%		
			\end{tabular}
		\end{table}
		%
		
		
		\textbf{High incidence setting (seasonal AR 16 per 100), scenario B.}
		\begin{table}
			\begin{tabular}{rcccc}
				$\gamma$ & VE new (\%) & VE prior & VE both & VE, adj.  (\%) \\
				\hline
				0.25 & 50 & -2.6 & 48.7 & 50 \\ 
				0.5 & 50 & -7.3 & 46.4 & 50 \\ 
				0.75 & 50 & -17 & 41.5 & 50 \\ 
				1 & 50 & -43.9 & 28.1 & 50 \\ 
				\hline
				%		
			\end{tabular}
		\end{table}
	}
\end{frame}
%
%
\begin{frame}
	{Could the apparently deleterious effect of repeated vaccinations on VE be due to confounding by vaccination history?}
	
	\centering
	\visible<2->{\Large \textbf{\textcolor{red}{Possibly, at least in part ?}}}
\end{frame}
%
\subsection{Revisiting Ohmit et al.}
\begin{frame}{Revisiting Ohmit et al.}
		\begin{figure}
		\includegraphics[width = \linewidth]{table1.png}
	\end{figure}
	\begin{itemize}
		\item Newly vaccinated should be compared with never vaccinated; this gives unadjusted VE of 61\%.
		\item Repeatedly vaccinated should be compared to those previously repeatedly, but not currently vaccinated: VE 30.7\% rather than 41\%.  Bias other than expected!
		
	\end{itemize}
\end{frame}
%
%
\section{Conclusions}

\begin{frame}{Conclusions and outlook}

{\scriptsize 	
	\begin{itemize}
	\item If the immunological consequences of vaccination and natural infection are different in terms of protection from infection, then \textbf{history of vaccination} could be a confounder of VE estimates.
	%
	\item Simulations studies show that under most, but the most extreme assumptions, the resulting bias in \emph{\textbf{crude}} VE estimates is\textbf{ relatively modest}.
	%
	\item Some VE estimates based on \emph{\textbf{categorical}} ``vaccination history'' covariate are substantially biased, even under moderate incidence setting.
	%
	\item Adequate adjustment by vaccination history will result in \textbf{valid VE estimates}; ``adequate adjustment'' refers to defining all immunologically distinct vaccination histories (vaccination up to previous season) and stratifying the analysis by these distinct histories.
	%
\end{itemize}}
\end{frame}
%
\begin{frame}
	\centering
		\visible<2->{\large \textbf{\textcolor{red}{Not Yet ...}}} 
		\LARGE \textbf{\textcolor{blue}{The End}}
		\visible<2->{\large \textbf{\textcolor{red}{...}}} 
\end{frame}
%
%
%	
\end{document}
