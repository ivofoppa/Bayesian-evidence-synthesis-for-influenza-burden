\documentclass[]{article}
\usepackage{lmodern}
\usepackage{amssymb,amsmath}
\usepackage{ifxetex,ifluatex}
\usepackage{fixltx2e} % provides \textsubscript
\ifnum 0\ifxetex 1\fi\ifluatex 1\fi=0 % if pdftex
  \usepackage[T1]{fontenc}
  \usepackage[utf8]{inputenc}
\else % if luatex or xelatex
  \ifxetex
    \usepackage{mathspec}
  \else
    \usepackage{fontspec}
  \fi
  \defaultfontfeatures{Ligatures=TeX,Scale=MatchLowercase}
\fi
% use upquote if available, for straight quotes in verbatim environments
\IfFileExists{upquote.sty}{\usepackage{upquote}}{}
% use microtype if available
\IfFileExists{microtype.sty}{%
\usepackage{microtype}
\UseMicrotypeSet[protrusion]{basicmath} % disable protrusion for tt fonts
}{}
\usepackage[margin=1in]{geometry}
\usepackage{hyperref}
\hypersetup{unicode=true,
            pdftitle={Bayesian evidence synthesis for influenza burden estimation, using FluSurv-Net data},
            pdfauthor={Ivo M. Foppa},
            pdfborder={0 0 0},
            breaklinks=true}
\urlstyle{same}  % don't use monospace font for urls
\usepackage{longtable,booktabs}
\usepackage{graphicx,grffile}
\makeatletter
\def\maxwidth{\ifdim\Gin@nat@width>\linewidth\linewidth\else\Gin@nat@width\fi}
\def\maxheight{\ifdim\Gin@nat@height>\textheight\textheight\else\Gin@nat@height\fi}
\makeatother
% Scale images if necessary, so that they will not overflow the page
% margins by default, and it is still possible to overwrite the defaults
% using explicit options in \includegraphics[width, height, ...]{}
\setkeys{Gin}{width=\maxwidth,height=\maxheight,keepaspectratio}
\IfFileExists{parskip.sty}{%
\usepackage{parskip}
}{% else
\setlength{\parindent}{0pt}
\setlength{\parskip}{6pt plus 2pt minus 1pt}
}
\setlength{\emergencystretch}{3em}  % prevent overfull lines
\providecommand{\tightlist}{%
  \setlength{\itemsep}{0pt}\setlength{\parskip}{0pt}}
\setcounter{secnumdepth}{0}
% Redefines (sub)paragraphs to behave more like sections
\ifx\paragraph\undefined\else
\let\oldparagraph\paragraph
\renewcommand{\paragraph}[1]{\oldparagraph{#1}\mbox{}}
\fi
\ifx\subparagraph\undefined\else
\let\oldsubparagraph\subparagraph
\renewcommand{\subparagraph}[1]{\oldsubparagraph{#1}\mbox{}}
\fi

%%% Use protect on footnotes to avoid problems with footnotes in titles
\let\rmarkdownfootnote\footnote%
\def\footnote{\protect\rmarkdownfootnote}

%%% Change title format to be more compact
\usepackage{titling}

% Create subtitle command for use in maketitle
\newcommand{\subtitle}[1]{
  \posttitle{
    \begin{center}\large#1\end{center}
    }
}

\setlength{\droptitle}{-2em}

  \title{Bayesian evidence synthesis for influenza burden estimation, using
FluSurv-Net data}
    \pretitle{\vspace{\droptitle}\centering\huge}
  \posttitle{\par}
    \author{Ivo M. Foppa}
    \preauthor{\centering\large\emph}
  \postauthor{\par}
      \predate{\centering\large\emph}
  \postdate{\par}
    \date{August 15, 2018}


\begin{document}
\maketitle

\subsection{Three models, compared to results provided by
Melissa}\label{three-models-compared-to-results-provided-by-melissa}

\begin{enumerate}
\def\labelenumi{\arabic{enumi}.}
\tightlist
\item
  The first model uses influenza detection probabilities, separate for
  patients hospitalized with and without fatal outcome, and standard
  errors estimates by Melissa to create Normal sampling distributions.
  The observed outcomes, hospitalization and death in hospitalized
  patients, are modeled as Poisson random variables generated from the
  (unobserved) total numbers of patients with that outcome and the
  provided detection probabilities. The latent (unobserved) outcome
  probabilities in FluSurv-NET participants is then used to generate US
  numbers.
\item
  The second model uses the provided sensitivity estimates/SE as priors
  for the test sensitivities, but uses the observed proportions
  positive, assuming influenza positivity to be independent of test
  type.
\item
  The third model relaxes the assumption of equal influenza positivity
  across test types.
\end{enumerate}

\subsection{Results}\label{results}

\subsubsection{Model conversion}\label{model-conversion}

\includegraphics{Methods_and_results_files/figure-latex/unnamed-chunk-1-1.pdf}
\includegraphics{Methods_and_results_files/figure-latex/unnamed-chunk-1-2.pdf}
\includegraphics{Methods_and_results_files/figure-latex/unnamed-chunk-1-3.pdf}

\subsubsection{Model estimates:
Comparison}\label{model-estimates-comparison}

The following Table compares Melissa's estimates to the estimates from
the three models

\begin{longtable}[]{@{}lllll@{}}
\toprule
\begin{minipage}[b]{0.12\columnwidth}\raggedright\strut
Outcome\strut
\end{minipage} & \begin{minipage}[b]{0.13\columnwidth}\raggedright\strut
Melissa\strut
\end{minipage} & \begin{minipage}[b]{0.12\columnwidth}\raggedright\strut
Model 1\strut
\end{minipage} & \begin{minipage}[b]{0.12\columnwidth}\raggedright\strut
Model 2\strut
\end{minipage} & \begin{minipage}[b]{0.12\columnwidth}\raggedright\strut
Model 3\strut
\end{minipage}\tabularnewline
\midrule
\endhead
\begin{minipage}[t]{0.12\columnwidth}\raggedright\strut
Hosp.s\strut
\end{minipage} & \begin{minipage}[t]{0.13\columnwidth}\raggedright\strut
445,724\strut
\end{minipage} & \begin{minipage}[t]{0.12\columnwidth}\raggedright\strut
4.44578\times 10\^{}\{5\}
(\texttt{rcihosp1{[}2{]}},\texttt{rcihosp1{[}3{]}})\strut
\end{minipage} & \begin{minipage}[t]{0.12\columnwidth}\raggedright\strut
3.1149\times 10\^{}\{5\}
(\texttt{rcihosp2{[}2{]}},\texttt{rcihosp2{[}3{]}})\strut
\end{minipage} & \begin{minipage}[t]{0.12\columnwidth}\raggedright\strut
4.47787\times 10\^{}\{5\}
(\texttt{rcihosp3{[}2{]}},\texttt{rcihosp3{[}3{]}})\strut
\end{minipage}\tabularnewline
\begin{minipage}[t]{0.12\columnwidth}\raggedright\strut
---------\strut
\end{minipage} & \begin{minipage}[t]{0.13\columnwidth}\raggedright\strut
----------\strut
\end{minipage} & \begin{minipage}[t]{0.12\columnwidth}\raggedright\strut
---------\strut
\end{minipage} & \begin{minipage}[t]{0.12\columnwidth}\raggedright\strut
---------\strut
\end{minipage} & \begin{minipage}[t]{0.12\columnwidth}\raggedright\strut
---------\strut
\end{minipage}\tabularnewline
\begin{minipage}[t]{0.12\columnwidth}\raggedright\strut
Deaths\strut
\end{minipage} & \begin{minipage}[t]{0.13\columnwidth}\raggedright\strut
41,743\strut
\end{minipage} & \begin{minipage}[t]{0.12\columnwidth}\raggedright\strut
6.1428\times 10\^{}\{4\}
(\texttt{rcideath1{[}2{]}},\texttt{rcideath1{[}3{]}})\strut
\end{minipage} & \begin{minipage}[t]{0.12\columnwidth}\raggedright\strut
4.4026\times 10\^{}\{4\}
(\texttt{rcideath2{[}2{]}},\texttt{rcideath2{[}3{]}})\strut
\end{minipage} & \begin{minipage}[t]{0.12\columnwidth}\raggedright\strut
1.01694\times 10\^{}\{5\}
(\texttt{rcideath3{[}2{]}},\texttt{rcideath3{[}3{]}})\strut
\end{minipage}\tabularnewline
\begin{minipage}[t]{0.12\columnwidth}\raggedright\strut
---------\strut
\end{minipage} & \begin{minipage}[t]{0.13\columnwidth}\raggedright\strut
----------\strut
\end{minipage} & \begin{minipage}[t]{0.12\columnwidth}\raggedright\strut
---------\strut
\end{minipage} & \begin{minipage}[t]{0.12\columnwidth}\raggedright\strut
---------\strut
\end{minipage} & \begin{minipage}[t]{0.12\columnwidth}\raggedright\strut
---------\strut
\end{minipage}\tabularnewline
\bottomrule
\end{longtable}


\end{document}
